\documentclass{article}
\usepackage[utf8]{inputenc}
\usepackage[T2A]{fontenc}
\usepackage[english, russian]{babel}

\usepackage{amsmath}
\usepackage{amssymb}
\usepackage{booktabs}
\usepackage{graphicx}
\usepackage{geometry}

\begin{document}

\title{Пример PDF с таблицей и изображением}
\author{Студент}
\maketitle


    \section{Введение}

    Этот документ демонстрирует возможности библиотеки \texttt{latexgen} для генерации LaTeX кода.

    \section{Таблица языков программирования}

    \begin{table}[h!]
    \centering
    \begin{tabular}{lcllc}
        {Язык} & {Год} & {Автор} & {Типизация} & {Уровень памяти} \\
        \hline
        {Python} & {1991} & {Гвидо ван Россум} & {Динамическая} & {Высокий} \\
        {Java} & {1995} & {Джеймс Гослинг} & {Статическая} & {Высокий} \\
        {JavaScript} & {1995} & {Брендан Эйх} & {Динамическая} & {Высокий} \\
        {Go} & {2009} & {Google} & {Статическая} & {Средний} \\
        {Rust} & {2010} & {Graydon Hoare} & {Статическая} & {Низкий} \\
        \hline
    \end{tabular}
    \caption{Сравнение современных языков программирования}
    \label{tab:languages}
\end{table}

    \section{Пример изображения}

    \begin{figure}[h!]
    \centering
    \includegraphics[width=0.6\textwidth,scale=0.8]{./hw_2/artifacts/sample_image.png}
    \caption{Пример вставки изображения в LaTeX документ}
    \label{fig:sample}
\end{figure}

    \section{Заключение}

    Документ был полностью сгенерирован с помощью Python библиотеки \texttt{latexgen}.
    Таблица и изображение добавлены автоматически.
    

\end{document}